\documentclass{article}
\usepackage{graphicx}
\usepackage[colorlinks=true,
  linkcolor=black,
  urlcolor=blue,
citecolor=black]{hyperref}
\usepackage{amsmath}
\usepackage{booktabs}
\usepackage{siunitx}
\usepackage{subcaption}
\usepackage{xcolor}
\usepackage{cite}
\usepackage[nottoc]{tocbibind}
\usepackage[english]{babel}
\usepackage{listings}

\tolerance=1413
\hfuzz=1.5pt

\renewcommand{\thesubsection}{\roman{subsection})} % chktex 9 chktex 10

% Code listing settings
\lstset{
basicstyle=\ttfamily\small,
breaklines=true,
frame=single,
numbers=left,
numberstyle=\tiny,
backgroundcolor=\color{gray!10}
}

\title{2MA918 Laboration I}
\author{Samuel Berg (sb224sc)}
\date{November 2025}

\begin{document}

\maketitle

\tableofcontents

\newpage
\section{Exercise 1: Warm-up LP-problem}

This exercise examines a television production optimization problem for a company producing two types of TVs (type A and type B) with limited resources across three production stages.

\subsection{Define the model}

The linear programming model is formulated as follows:

\textbf{Decision variables:}
\begin{itemize}
\item $x_1$: Number of type A TVs to produce
\item $x_2$: Number of type B TVs to produce
\end{itemize}

\textbf{Objective function:}
\begin{equation}
\text{maximize } z = 700x_1 + 1000x_2
\end{equation}

\textbf{Constraints:}
\begin{align}
3x_1 + 5x_2 &\leq 3900 \quad \text{(Stage I)} \\
x_1 + 3x_2 &\leq 2100 \quad \text{(Stage II)} \\
2x_1 + 2x_2 &\leq 2200 \quad \text{(Stage III)} \\
x_1, x_2 &\geq 0
\end{align}

In matrix form: maximize $c^T x$ subject to $Ax \leq b$, $x \geq 0$, where:
\begin{equation}
c =
\begin{bmatrix} 700 \\ 1000
\end{bmatrix}, \quad
A =
\begin{bmatrix} 3 & 5 \\ 1 & 3 \\ 2 & 2
\end{bmatrix}, \quad
b =
\begin{bmatrix} 3900 \\ 2100 \\ 2200
\end{bmatrix}
\end{equation}

\subsection{Restricted convex set region~\label{subsec:convex-region}}

The feasible region is bounded by the constraint lines and the non-negativity conditions. Figure~\ref{fig:feasible_region} shows the three constraint lines along with the $x_1$ and $x_2$ axes.

\begin{figure}[!ht]
\centering
\includegraphics[width=0.8\textwidth]{img/ex1_feasible_region.png}
\caption{Feasible region for the TV production problem}\label{fig:feasible_region}
\end{figure}

\subsection{Study the figure of~\ref{subsec:convex-region}}

From the figure, we can identify five vertices that enclose the feasible region:
\begin{itemize}
\item $v_1 = (0, 0)$ --- Origin
\item $v_2 = (1100, 0)$ --- Intersection of $x_2 = 0$ and Stage III constraint
\item $v_3 = (800, 300)$ --- Intersection of Stage I and Stage III constraints
\item $v_4 = (300, 600)$ --- Intersection of Stage I and Stage II constraints
\item $v_5 = (0, 700)$ --- Intersection of $x_1 = 0$ and Stage II constraint
\end{itemize}

\subsection{Adding level curves to convex hull plot}

Figure~\ref{fig:level_curves} shows the feasible region with level curves of the objective function superimposed. The level curves are lines of constant profit $z = 700x_1 + 1000x_2 = k$ for increasing values of $k$. The curves have increasing red intensity for higher profit values. The maximum profit is achieved at the vertex where the highest level curve touches the feasible region, which is $v_3 = (800, 300)$.

\begin{figure}[!ht]
\centering
\includegraphics[width=0.8\textwidth]{img/ex1_level_curves.png}
\caption{Feasible region with level curves of the objective function}\label{fig:level_curves}
\end{figure}

\subsection{Verify in all extreme points}

Evaluating the objective function at each vertex:
\begin{align*}
z(v_1) &= 700(0) + 1000(0) = 0 \\
z(v_2) &= 700(1100) + 1000(0) = 770{,}000 \\
z(v_3) &= 700(800) + 1000(300) = 860{,}000 \\
z(v_4) &= 700(300) + 1000(600) = 810{,}000 \\
z(v_5) &= 700(0) + 1000(700) = 700{,}000
\end{align*}

The maximum value is $z^* = 860{,}000$ at vertex $v_3 = (800, 300)$.

\subsection{Find the maximum}

Using \texttt{scipy.optimize.linprog} with the HiGHS method to solve the problem:
\begin{itemize}
\item Optimal solution: $x^* = (800, 300)$
\item Maximum profit: $z^* = 860{,}000$
\end{itemize}

This confirms the result from the graphical analysis. The company should produce 800 type A TVs and 300 type B TVs to maximize profit at \$860,000.

\subsection{Problem in \textit{standard form}}

Converting to standard form by introducing slack variables $s_1, s_2, s_3 \geq 0$:
\begin{align}
3x_1 + 5x_2 + s_1 &= 3900 \\
x_1 + 3x_2 + s_2 &= 2100 \\
2x_1 + 2x_2 + s_3 &= 2200 \\
x_1, x_2, s_1, s_2, s_3 &\geq 0
\end{align}

Solving in standard form yields:
\begin{itemize}
\item Optimal solution: $x^* = (800, 300)$
\item Slack variables: $s^* = (0, 400, 0)$
\item Maximum profit: $z^* = 860{,}000$
\end{itemize}

The slack variable $s_2 = 400$ indicates that Stage II has 400 hours of unused capacity. Stage I and Stage III are fully utilized (slack = 0), making them binding constraints.

\newpage
\section{Exercise 2: Large LP-problems}

This exercise investigates the computational performance of different optimization methods on large-scale linear programming problems.

\subsection{Matrix and vectors solution, why?}

Matrix and vector notation provides several critical advantages for large LP problems:
\begin{enumerate}
\item \textbf{Compact representation:} Instead of writing thousands of individual equations, we represent them as $Ax \leq b$.
\item \textbf{Computational efficiency:} Matrix operations are highly optimized in numerical libraries (e.g., BLAS, LAPACK), enabling fast computation.
\item \textbf{Scalability:} The same algorithm handles problems of any size without modification.
\item \textbf{Mathematical clarity:} The structure of the problem is clear, facilitating theoretical analysis.
\item \textbf{Implementation simplicity:} Reduces code complexity and potential errors when dealing with large systems.
\end{enumerate}

\subsection{Solving example with simplex}

For a small test problem with $m = n = 10$, the simplex method solves the problem in approximately 5--6 milliseconds on the test machine. This demonstrates that for small problems, the simplex method is quite efficient.

\subsection{Get simplex average time exceeds 1 second}

By systematically increasing the problem size and averaging over 5 trials, we find that the simplex method exceeds 1 second average time at:
\begin{equation}
\boxed{m = n = 150}
\end{equation}

The computational cost grows approximately cubically with problem size, as expected for the simplex method.

\subsection{Replace simplex with more sophisticated method}

Using the HiGHS method (a state-of-the-art interior point algorithm), we find that the average time exceeds 1 second at:
\begin{equation}
\boxed{m = n = 1400}
\end{equation}

This represents approximately a \textbf{9.3× performance improvement} over the simplex method. Figure~\ref{fig:performance} shows the comparison between both methods.

\begin{figure}[!ht]
\centering
\includegraphics[width=\textwidth]{img/ex2_performance_comparison.png}
\caption{Performance comparison: Simplex vs HiGHS method}\label{fig:performance}
\end{figure}

The HiGHS method demonstrates superior scalability for large problems, making it the preferred choice for industrial-scale optimization tasks.

\newpage
\section{Exercise 3: Sensitivity analysis}

This exercise performs sensitivity analysis on the TV production problem from Exercise 1 to understand how changes in parameters affect the optimal solution.

\subsection{Solve the dual problem and verify}

The dual problem is formulated as:
\begin{align}
\text{minimize } \quad & 3900y_1 + 2100y_2 + 2200y_3 \\
\text{subject to} \quad & 3y_1 + y_2 + 2y_3 \geq 700 \\
& 5y_1 + 3y_2 + 2y_3 \geq 1000 \\
& y_1, y_2, y_3 \geq 0
\end{align}

Solving the dual problem yields:
\begin{itemize}
\item Dual optimal solution: $y^* = (150, 0, 125)$
\item Dual optimal value: $w^* = 860{,}000$
\end{itemize}

\textbf{Verification:} The primal optimal value equals the dual optimal value ($z^* = w^* = 860{,}000$), confirming strong duality.

\subsection{Shadow price of various constraints}

The shadow prices (dual variables) are:
\begin{itemize}
\item Stage I ($y_1$): 150
\item Stage II ($y_2$): 0
\item Stage III ($y_3$): 125
\end{itemize}

\textbf{Interpretation:}
\begin{itemize}
\item Stage I has shadow price 150, meaning each additional hour increases profit by \$150.
\item Stage II has shadow price 0, indicating it has slack capacity (non-binding constraint).
\item Stage III has shadow price 125, meaning each additional hour increases profit by \$125.
\end{itemize}

\subsection{100 extra working hours}

Testing the addition of 100 hours to each stage:
\begin{center}
\begin{tabular}{lccc}
  \toprule
  Stage & Shadow Price & Predicted Increase & Actual Increase \\
  \midrule
  Stage I & 150 & \$15,000 & \$15,000 \\
  Stage II & 0 & \$0 & \$0 \\
  Stage III & 125 & \$12,500 & \$12,500 \\
  \bottomrule
\end{tabular}
\end{center}

\textbf{Recommendation:}
\begin{itemize}
\item \textbf{INVEST} in Stage I (highest shadow price of 150)
\item \textbf{DO NOT invest} in Stage II (shadow price of 0, has excess capacity)
\item Stage III is also valuable (shadow price of 125)
\end{itemize}

\subsection{Price increase to change optimal solution}

Testing price increases for type B TVs, we find that the optimal solution changes when the price increases by:
\begin{equation}
\boxed{\Delta p_B = \$200}
\end{equation}

At price $p_B = \$1200$ (original \$1000 + \$200), the optimal solution shifts from $(800, 300)$ to $(300, 600)$, indicating a switch to producing more type B TVs.

\subsection{New TV, should be produced or not?}

Type C TV specifications:
\begin{itemize}
\item Profit: \$1,350
\item Production times: $(7, 4, 2)$ hours for stages I, II, III
\end{itemize}

The reduced cost for type C is:
\begin{equation}
r_C = c_C - y^* \cdot a_C = 1350 - (150 \times 7 + 0 \times 4 + 125 \times 2) = 1350 - 1300 = 50
\end{equation}

Since the reduced cost is positive ($r_C = 50 > 0$), \textbf{type C should be PRODUCED}.

Solving the extended problem with type C included:
\begin{itemize}
\item Optimal production: $(950, 0, 150)$ units of types A, B, and C
\item New optimal profit: \$867,500
\end{itemize}

This confirms that producing type C increases profit by \$7,500.

\subsection{Quality inspection working time increase}

Quality inspection times:
\begin{itemize}
\item Type A\@: 0.5 hours
\item Type B\@: 0.75 hours
\item Type C\@: 0.1 hours
\end{itemize}

With optimal production of $(950, 0, 150)$ units:
\begin{equation}
\text{Total inspection hours} = 950 \times 0.5 + 0 \times 0.75 + 150 \times 0.1 = 475 + 0 + 15 = 490 \text{ hours}
\end{equation}

\textbf{Answer:} The company must add \boxed{490 \text{ hours}} to the quality inspection line to maintain current production levels without interference.

% Refrences
% \newpage
% \bibliographystyle{ieeetr}
% \bibliography{ref}

% Appendix
% \newpage
% \pagenumbering{Alph}
% \setcounter{page}{1}

\end{document} % chktex 17
